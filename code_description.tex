% Options for packages loaded elsewhere
\PassOptionsToPackage{unicode}{hyperref}
\PassOptionsToPackage{hyphens}{url}
%
\documentclass[
]{article}
\usepackage{amsmath,amssymb}
\usepackage{iftex}
\ifPDFTeX
  \usepackage[T1]{fontenc}
  \usepackage[utf8]{inputenc}
  \usepackage{textcomp} % provide euro and other symbols
\else % if luatex or xetex
  \usepackage{unicode-math} % this also loads fontspec
  \defaultfontfeatures{Scale=MatchLowercase}
  \defaultfontfeatures[\rmfamily]{Ligatures=TeX,Scale=1}
\fi
\usepackage{lmodern}
\ifPDFTeX\else
  % xetex/luatex font selection
\fi
% Use upquote if available, for straight quotes in verbatim environments
\IfFileExists{upquote.sty}{\usepackage{upquote}}{}
\IfFileExists{microtype.sty}{% use microtype if available
  \usepackage[]{microtype}
  \UseMicrotypeSet[protrusion]{basicmath} % disable protrusion for tt fonts
}{}
\makeatletter
\@ifundefined{KOMAClassName}{% if non-KOMA class
  \IfFileExists{parskip.sty}{%
    \usepackage{parskip}
  }{% else
    \setlength{\parindent}{0pt}
    \setlength{\parskip}{6pt plus 2pt minus 1pt}}
}{% if KOMA class
  \KOMAoptions{parskip=half}}
\makeatother
\usepackage{xcolor}
\usepackage[margin=1in]{geometry}
\usepackage{graphicx}
\makeatletter
\def\maxwidth{\ifdim\Gin@nat@width>\linewidth\linewidth\else\Gin@nat@width\fi}
\def\maxheight{\ifdim\Gin@nat@height>\textheight\textheight\else\Gin@nat@height\fi}
\makeatother
% Scale images if necessary, so that they will not overflow the page
% margins by default, and it is still possible to overwrite the defaults
% using explicit options in \includegraphics[width, height, ...]{}
\setkeys{Gin}{width=\maxwidth,height=\maxheight,keepaspectratio}
% Set default figure placement to htbp
\makeatletter
\def\fps@figure{htbp}
\makeatother
\setlength{\emergencystretch}{3em} % prevent overfull lines
\providecommand{\tightlist}{%
  \setlength{\itemsep}{0pt}\setlength{\parskip}{0pt}}
\setcounter{secnumdepth}{-\maxdimen} % remove section numbering
\ifLuaTeX
  \usepackage{selnolig}  % disable illegal ligatures
\fi
\IfFileExists{bookmark.sty}{\usepackage{bookmark}}{\usepackage{hyperref}}
\IfFileExists{xurl.sty}{\usepackage{xurl}}{} % add URL line breaks if available
\urlstyle{same}
\hypersetup{
  pdftitle={Code Description for reviewers},
  pdfauthor={Se Yoon Lee, Peng Zhao, Debdeep Pati, and Bani Mallick},
  hidelinks,
  pdfcreator={LaTeX via pandoc}}

\title{Code Description for reviewers}
\author{Se Yoon Lee, Peng Zhao, Debdeep Pati, and Bani Mallick}
\date{}

\begin{document}
\maketitle

\hypertarget{about-codes-smalltextsfhorseshoe_van_der_pas...-and-smalltextsfglt_prior}{%
\section{\texorpdfstring{1. About codes
\(\small{\textsf{horseshoe_van_der_pas}(...)}\) and
\(\small{\textsf{GLT_prior}}\)}{1. About codes \textbackslash small\{\textbackslash textsf\{horseshoe\_van\_der\_pas\}(...)\} and \textbackslash small\{\textbackslash textsf\{GLT\_prior\}\}}}\label{about-codes-smalltextsfhorseshoe_van_der_pas...-and-smalltextsfglt_prior}}

The objective of this html document is to describe some guidelines for
users to execute R functions \(\textsf{horseshoe_van_der_pas}(...)\) and
\(\textsf{GLT_prior}(...)\). Both R functions aim to realize samples via
a Markov Chain Monte Carlo algorithm from the unknown parameter
\(\beta \in \mathbb{R}^p\) under a high-dimensional linear regression
\[  \textbf{y}=\textbf{X} \beta + \sigma\epsilon, \quad \epsilon \sim \mathcal{N}_{n}(\textbf{0},\textbf{I}_n), \quad \textbf{X} \in \mathbb{R}^{n \times p},\]
where the prior distribution for the \(\beta\) is given by either the
Horseshoe prior (\(\textsf{horseshoe_van_der_pas}(...)\)) or the GLT
prior (\(\textsf{GLT_prior}(...)\)). The prior for the measurement error
\(\sigma^2\) is given by the Jeffreys prior
\(\sigma^2 \sim \pi(\sigma^2) \propto 1/\sigma^2\) for the both priors.

The hierarchies of the Horseshoe and GLT priors are given as:

\[
\Large
\textbf{A hierarchy of the Horseshoe}
\] The Horseshoe is given by

\[\beta_j|\lambda_j,\tau, \sigma^2 \sim \mathcal{N}_{1}(0,\lambda_j^2 \tau^2 \sigma^2), \quad (j=1,\cdots, p),\]

\[\lambda_j, \tau \sim \mathcal{C}^{+}(0,1), \quad (j=1,\cdots, p),\]

To draw samples from the posterior, we use the code
\(\textsf{horseshoe_van_der_pas}(...)\) which is extracted from the CRAN
package
\url{https://cran.r-project.org/web/packages/horseshoe/horseshoe.pdf} as
follows:

\[\textsf{horseshoe_van_der_pas}\text{(y, X, method.tau = "halfCauchy", method.sigma = "Jeffreys", burn, nmc, thin)}\]
where \(\text{y}=\textbf{y}\) and \(\text{X}=\textbf{X}\).

\[
\Large
\textbf{A hierarchy of the GLT prior}
\]

The GLT prior is given by

\[\beta_j|\lambda_j, \sigma^2 \sim \mathcal{N}_{1}(0,\lambda_j^2 \sigma^2), \quad (j=1,\cdots, p),\]

\[\lambda_j|\tau, \xi \sim \mathcal{GPD}(\tau, \xi), \quad (j=1,\cdots, p),\]

\[\tau|\xi \sim \mathcal{IG}(p/\xi + 1,1),\quad 
\xi \sim \log\ \mathcal{N}(\mu, \rho^2), \quad \mu \in \mathbb{R}, \quad\rho^2 >0.
\] The \(\rho^2\) and \(\mu\) are hyper-parameters. We use
\(\rho^2=0.001\) as a default value for \(\rho^2\), while the \(\mu\) is
calibrated by `Elliptical slice sampler centered by the Hill estimator';
refer to Algorithm 1 in Supplemental Material. Eventually, users do not
need any expert-tuning for the hyper-parameters.

To draw samples from the posterior distribution, one should use
\[\textsf{GLT_prior}\text{(y, X, BCM_sampling = c(TRUE,FALSE), burn, nmc, thin)}\]
where \(\text{y}=\textbf{y}\) and \(\text{X}=\textbf{X}\).

\hypertarget{common-inputs-for-the-two-codes}{%
\section{2. Common inputs for the two
codes}\label{common-inputs-for-the-two-codes}}

The followings are common inputs for the two codes
\(\textsf{horseshoe_van_der_pas}(...)\) and \(\textsf{GLT_prior}(...)\).

\(\text{y}:\) \(n\)-dimensional response vector
(\(\text{y} = \textbf{y}\))

\(\text{X}:\) \(n\)-by-\(p\) design matrix (\(\text{X} = \textbf{X}\))

\(\text{nmc}:\) number of mcmc samples after burning

\(\text{burn}:\) number of burned samples

\(\text{thin}:\) thinning number for the mcmc samples after burning

\hypertarget{more-about-the-code-smalltextsfhorseshoe_van_der_pas...}{%
\section{\texorpdfstring{3. More about the code
\(\small{\textsf{horseshoe_van_der_pas}(...)}\)}{3. More about the code \textbackslash small\{\textbackslash textsf\{horseshoe\_van\_der\_pas\}(...)\}}}\label{more-about-the-code-smalltextsfhorseshoe_van_der_pas...}}

The code \(\textsf{horseshoe_van_der_pas}(...)\) is the same as the
function \(\textsf{horseshoe}(...)\) in the \(\textsf{R}\) package
\(\textbf{horseshoe}\) except that
\(\textsf{horseshoe_van_der_pas}(...)\) can additionally print out the
posterior samples of local-scale parameters
\(\lambda_{j} (j=1,\cdots, p)\). Refer to the manual of the package for
more details.

\hypertarget{more-about-the-code-smalltextsfglt_prior...}{%
\section{\texorpdfstring{4. More about the code
\(\small{\textsf{GLT_prior}(...)}\)}{4. More about the code \textbackslash small\{\textbackslash textsf\{GLT\_prior\}(...)\}}}\label{more-about-the-code-smalltextsfglt_prior...}}

The code \(\textsf{GLT_prior}(...)\) allows the user to choose what
technique to use to sample from the full-conditional posterior
distribution for \(\beta\) which is given by
\[\pi(\beta|-) \sim \mathcal{N}_{p}(\Sigma\textbf{X}^{\top}\textbf{y},\sigma^2\Sigma), \quad \Sigma = (\textbf{X}^{\top}\textbf{X} + \Lambda^{-1})^{-1} \in \mathbb{R}^{p\times p}\]
where
\(\Lambda = \text{diag}(\lambda_1^2,\cdots, \lambda_p^2)\in \mathbb{R}^{p \times p}\)
(See the Step 1 of Subsection A.1 in Supplementary material).

If \(\text{BCM_sampling = TRUE}\) is selected, then the code
\(\textsf{GLT_prior}(...)\) uses the fast-sampling algorithm developed
by Bhattacharya et al (Biometrika, 2016) to sample from the
\(\pi(\beta|-)\) at each iteration. Otherwise, if selecting
\(\text{BCM_sampling = FALSE}\), then the code
\(\textsf{GLT_prior}(...)\) uses an algorithm developed by Rue (JRSSB,
2001).

\hypertarget{output-for-the-code-smalltextsfglt_prior}{%
\section{\texorpdfstring{5. OUTPUT for the code
\(\small{\textsf{GLT_prior}}\)}{5. OUTPUT for the code \textbackslash small\{\textbackslash textsf\{GLT\_prior\}\}}}\label{output-for-the-code-smalltextsfglt_prior}}

Output of the code \(\textsf{GLT_prior}\) is as follows:

\(\text{beta.vec:}\) posterior realizations of coefficients
\(\beta = (\beta_1, \cdots, \beta_p)^{\top} \in \mathbb{R}^p\)

\(\text{lambda.vec:}\) posterior realizations of \(p\) local-scale
parameters
\(\lambda = (\lambda_1, \cdots, \lambda_p)^{\top} \in \mathbb{R}^p\)

\(\text{tau:}\) posterior realizations of global-scale parameter
\(\tau\)

\(\text{sigma.sq:}\) posterior realizations of measurement error
\(\sigma^2\)

\(\text{xi:}\) posterior realizations of shape parameter \(\xi\)

\end{document}
